\documentclass{article}

\usepackage{fancyhdr} % Required for custom headers
\usepackage{lastpage} % Required to determine the last page for the footer
\usepackage{extramarks} % Required for headers and footers
\usepackage[usenames,dvipsnames]{color} % Required for custom colors
\usepackage{graphicx} % Required to insert images
\usepackage{listings} % Required for insertion of code
\usepackage{courier} % Required for the courier font
\usepackage{caption}
\usepackage{multirow}
\usepackage{subcaption}
\renewcommand{\_}{\char`_}
\renewcommand{\tt}{\lstinline}

% Margins
\topmargin=-0.45in
\evensidemargin=0in
\oddsidemargin=0in
\textwidth=6.5in
\textheight=9.0in
\headsep=0.25in

\linespread{1.1} % Line spacing

\lstset{language=C,
                basicstyle=\ttfamily,
                keywordstyle=\color{blue}\ttfamily,
                morekeywords={bool},
                stringstyle=\color{red}\ttfamily,
                commentstyle=\color{Plum}\ttfamily,
                morecomment=[l][\color{magenta}]{\#}
}


% Set up the header and footer
\pagestyle{fancy}
\lhead{Group 18} % Top left header
\chead{Task 1: Scheduling} % Top center head
\rhead{\firstxmark} % Top right header
\lfoot{\lastxmark} % Bottom left footer
\rfoot{Page\ \thepage\ of\ \protect\pageref{LastPage}} % Bottom right footer
\renewcommand\headrulewidth{0.4pt} % Size of the header rule
\renewcommand\footrulewidth{0.4pt} % Size of the footer rule

\setlength\parindent{0pt} % Removes all indentation from paragraphs

%----------------------------------------------------------------------------------------
%	TITLE PAGE
%----------------------------------------------------------------------------------------

\title{
\vspace{2in}
\textmd{\textbf{Task 2: User Programs}}\\
\normalsize\vspace{0.1in}\small{Due\ on\ Tuesday,\ March\ 01,\ 2016}\\
\vspace{0.1in}\large{\textbf{Pintos Group 18}}
\vspace{3in}
}

\author{Corentin Herbinet, Ignacio Navarro, William Springsteen}
\date{}

%----------------------------------------------------------------------------------------

\begin{document}

\maketitle
\newpage

\section{Design Questions: Argument Passing}
\section{Design Questions: System Calls}
\subsection{Data Structures}
\subsubsection{Purpose of new variables and \texttt{struct} members}
\begin{enumerate}

\item \begin{lstlisting}
struct thread
  {
		.
		.
		
#ifdef USERPROG
    uint32_t *pagedir;              /* Page directory. */
    struct list children;
    struct list_elem child_elem;  
    bool waited_on;        
    struct semaphore exit_sema;  
    struct semaphore load_sema;  
    bool loaded;
    char executable[MAX_FILENAME_LENGTH]; 
    int exit_status;
    struct list files;        
    int next_file_descriptor;  
#endif
		.
		.
  };
\end{lstlisting}

\item \begin{lstlisting}
/* Owned by userprog/syscall.h. */
struct proc_file {
  struct file *file;
  int fd;
  struct list_elem file_elem;
};
\end{lstlisting}

\item \begin{lstlisting}
/* Owned by userprog/syscall.c. */
struct lock secure_file;
\end{lstlisting}

\end{enumerate}

\begin{enumerate}
\item In the \texttt{struct thread} we have added:
\begin{itemize}
\item \lstinline{struct list children}: List of child threads this thread has. 
\item \lstinline{struct list_elem child_elem}: A struct \lstinline{list_elem} to be put in the list of another thread's children. 
\item \lstinline{bool waited_on}: Boolean set to true if the thread's parent has waited on this thread. Note we never set to false after 
being set to true, because we do not want to be able to wait on the same thread twice.
\item \lstinline{struct semaphore exit_sema}: Semaphore initialised to 0 to ensure the wait system call will wait until the thread has exited.
\item \lstinline{struct semaphore load_sema}: Semaphore to ensure the exec system call does not check to see if the child has successfully loaded until it has tried to be loaded.
\item \lstinline{bool loaded}: Set to the return value of \lstinline{load()} in \lstinline{start_process()}, so that \lstinline{sys_exec()} can check whether the child loaded successfully or not, as if not, -1 should be returned from \lstinline{sys_exec()}. \lstinline{loaded} is set to false when the thread is created. 
\item \lstinline{char executable[MAX_FILENAME_LENGTH]}: In \tt{start_process()}, if we load an executable on a thread,  \lstinline{executable} will be set to the filename of the executable. This will be useful in denying and allowing writes. 
\item \lstinline{int exit_status}: Status when the system exits.
\item \lstinline{struct list files}: List of files that a thread has open (Same file can be open with different fd).
\item \lstinline{int next_file_descriptor}: Next file to be opened by this process/thread will take this as its file descriptor. Incremented after a file is opened.
\end{itemize}
\item Each thread (i.e. process, as Pintos is not multithreaded) has a list of \texttt{proc\_files} to represent the file descriptors it has open. Two different \texttt{proc\_files} (even open in the same process) can have the same file member, but a different fd, due to it being opened twice. 
\item A lock used to ensure synchronization when accessing or modifying files.

\end{enumerate}


\subsubsection{File descriptors}
Each open file in a process has a file descriptor that represents the file. In other words, file descriptors index the open files in a process. In our implementation, we have a \texttt{proc\_file} that has a file descriptor and the associated file. When we are given a file descriptor to access a file, we iterate over the list of \texttt{proc\_file} a thread owns stopping when the file descriptors match.
File descriptors are unique within a single process, but not within the entire OS since they only index open files within a process, i.e, two different processes may have same file descriptors but they work independently.






%----------------------------------------------------------------------------------------

\end{document}
