\documentclass{article}

\usepackage{fancyhdr} % Required for custom headers
\usepackage{lastpage} % Required to determine the last page for the footer
\usepackage{extramarks} % Required for headers and footers
\usepackage[usenames,dvipsnames]{color} % Required for custom colors
\usepackage{graphicx} % Required to insert images
\usepackage{listings} % Required for insertion of code
\usepackage{courier} % Required for the courier font
\usepackage{caption}
\usepackage{multirow}
\usepackage{subcaption}
\renewcommand{\_}{\char`_}
\renewcommand{\tt}{\lstinline}

% Margins
\topmargin=-0.45in
\evensidemargin=0in
\oddsidemargin=0in
\textwidth=6.5in
\textheight=9.0in
\headsep=0.25in

\linespread{1.1} % Line spacing

\lstset{language=C,
                basicstyle=\ttfamily,
                keywordstyle=\color{blue}\ttfamily,
                morekeywords={bool},
                stringstyle=\color{red}\ttfamily,
                commentstyle=\color{Plum}\ttfamily,
                morecomment=[l][\color{magenta}]{\#}
}


% Set up the header and footer
\pagestyle{fancy}
\lhead{Group 18} % Top left header
\chead{Task 2: User Programs} % Top center head
\rhead{\firstxmark} % Top right header
\lfoot{\lastxmark} % Bottom left footer
\rfoot{Page\ \thepage\ of\ \protect\pageref{LastPage}} % Bottom right footer
\renewcommand\headrulewidth{0.4pt} % Size of the header rule
\renewcommand\footrulewidth{0.4pt} % Size of the footer rule

\setlength\parindent{0pt} % Removes all indentation from paragraphs

%----------------------------------------------------------------------------------------
%	TITLE PAGE
%----------------------------------------------------------------------------------------

\title{
\vspace{2in}
\textmd{\textbf{Task 2: User Programs}}\\
\normalsize\vspace{0.1in}\small{Due\ on\ Tuesday,\ March\ 01,\ 2016}\\
\vspace{0.1in}\large{\textbf{Pintos Group 18}}
\vspace{3in}
}

\author{Corentin Herbinet, Ignacio Navarro, William Springsteen}
\date{}

%----------------------------------------------------------------------------------------

\begin{document}

\maketitle
\newpage

\section{Design Questions: Argument Passing}

\subsection{Purpose of new Data Structures}

\begin{itemize}
\item \begin{lstlisting}
struct process_info {
  char *filename;
  int   argc;
  char *argv[];
};
\end{lstlisting}

\item This new struct represents the information linked to a process: its file name (a string), its number of arguments (an int) and a list of arguments (represented by an array of strings). 
\end{itemize}


\subsection{Algorithms: Implementation of Argument Parsing}

The following steps are taken in order to parse the file's arguments:

\begin{itemize}

\item We allocate memory for a new struct \texttt{process\_info} by using the \texttt{palloc\_get\_page()} method.
\item We create a copy of the given unparsed, const string by using the \texttt{memcpy()} method, after having checked that we have enough space in memory to do so. This is done because the \texttt{strtok\_r} method needs a modifiable string to tokenise.
\item We then use the \texttt{strtok\_r()} method in a loop to tokenise the given string into arguments each time we encounter a space character. If the \texttt{NULL} token is found, this means we have reached the end of the string and therefore all the arguments have been parsed. We store each token in the \texttt{argv} array of our \texttt{process\_info} struct and we increment \texttt{argc} in each iteration of the loop to reflect the number of tokens and therefore of arguments. We also check in every iteration of the tokenising loop that we have enough space in memory to store the next tokenised argument in our \texttt{process\_info} struct.

\end{itemize}


We then pass our \texttt{process\_info} struct to the \texttt{start\_process()} method so it can access the number of arguments and the arguments themselves. The following steps are taken to ensure the arguments are correctly set up on the stack:

\begin{itemize}

\item We find the address of the bottom of the stack page, using the stack pointer that is set at the top of the stack page. This will allow us to make sure the stack doesn't oveflow. We will indeed check that the stack pointer's position is not lower that the bottom of the stack page, everytime we push a string or a pointer to the stack.
\item We push the process's arguments onto the stack, in reverse order, and then recycle the \texttt{argv} field of our \texttt{process\_info} struct to store pointers to the arguments, which will be pushed to the stack later.
\item We round down the stack pointer to the nearest word multiple for faster access to the arguments.
\item We push a null pointer sentinel, followed by the pointers to the arguments, in reverse order as the stack grows downwards. 
\item Finally, we push a pointer to the first argument pointer, followed by the number of arguments \texttt{argc} and a fake return address.
\item If a stack overflow occurs or the process does not load properly, the thread exits.

\end{itemize}


\subsection{Rationale}

\subsubsection{Implementing \texttt{strtok\_r} but not \texttt{strtok}}

The standard \texttt{strtok()} method is not implemented as it is unsafe to use in threaded programs such as kernels. It indeed uses global data: it stores intermediate variables globally (specifically, a static buffer while parsing) which are reused at each call. We therefore need a re-entrant function, which is a function that can be invoked while already in the process of executing. The \texttt{strtok\_r()} method offers this by storing its data locally in a modifiable string, which makes it safe to use in threaded programs such as kernels.

\subsubsection{Advantages of separating command with the shell}

Separating commands into process name and arguments in the shell, as is implemented in the UNIX operating system, offers multiple advantages when we compare it to separating the commands in the kernel as is implemented in the Pintos operating system.

Firstly, it is more secure as the user typically should never directly interact with the kernel. Indeed, the kernel should never be accessed by the user by using abstraction as it is only responsible for starting and executing system processes. The shell, however, is a layer of abstraction created for the user to communicate with the kernel and ask it to run certain processes when needed. For instance, the shell is an utility that will check for any unsafe commands before communicating with the kernel itself. It does not have any system privileges, but once the kernel is called then it has access to all system processes.

Secondly, it offers better performance. Indeed, in cases of redirection in shell operations, such as when the program invoked is a shell command, the shell can execute it directly without having to communicate with the kernel. If the command is not a shell command, then the kernel is invoked to execute the program. This allows for better performance and speed in the execution of programs.


\section{Design Questions: System Calls}
\subsection{Data Structures}
\subsubsection{Purpose of new variables and \texttt{struct} members}
\begin{enumerate}

\item \begin{lstlisting}
struct thread
  {
		.
		.
		
#ifdef USERPROG
    uint32_t *pagedir;              /* Page directory. */
    struct list children;
    struct list_elem child_elem;  
    bool waited_on;        
    struct semaphore exit_sema;
    struct semaphore before_exit_sema; 
    struct semaphore load_sema;  
    bool loaded;
    char executable[MAX_FILENAME_LENGTH]; 
    int exit_status;
    struct list files;        
    int next_file_descriptor;  
#endif
		.
		.
  };
\end{lstlisting}

\item \begin{lstlisting}
/* Owned by userprog/syscall.h. */
struct proc_file {
  struct file *file;
  int fd;
  struct list_elem file_elem;
};
\end{lstlisting}

\item \begin{lstlisting}
/* Owned by userprog/syscall.c. */
struct lock secure_file;
\end{lstlisting}

\end{enumerate}

\begin{enumerate}
\item In the \texttt{struct thread} we have added:
\begin{itemize}
\item \lstinline{struct list children}: List of child threads this thread has. 
\item \lstinline{struct list_elem child_elem}: A struct \lstinline{list_elem} to be put in the list of another thread's children. 
\item \lstinline{bool waited_on}: Boolean set to true if the thread's parent has waited on this thread. Note we never set to false after 
being set to true, because we do not want to be able to wait on the same thread twice.
\item \lstinline{struct semaphore exit_sema}: Semaphore initialised to 0 to ensure the wait system call will wait until the thread has exited.
\item \lstinline{struct semaphore before_exit_sema}: Semaphore initialised to 0 to ensure that a thread cannot exit until we have obtained its exit status in a wait system call.
\item \lstinline{struct semaphore load_sema}: Semaphore to ensure the exec system call does not check to see if the child has successfully loaded until it has tried to be loaded.
\item \lstinline{bool loaded}: Set to the return value of \lstinline{load()} in \lstinline{start_process()}, so that \lstinline{sys_exec()} can check whether the child loaded successfully or not, as if not, -1 should be returned from \lstinline{sys_exec()}. \lstinline{loaded} is set to false when the thread is created. 
\item \lstinline{char executable[NAME_MAX]}: In \tt{start_process()}, if we load an executable on a thread,  \lstinline{executable} will be set to the filename of the executable. This will be useful in denying and allowing writes. 
\item \lstinline{int exit_status}: Status when the system exits.
\item \lstinline{struct list files}: List of files that a thread has open (Same file can be open with different fd).
\item \lstinline{int next_file_descriptor}: Next file to be opened by this process/thread will take this as its file descriptor. Incremented after a file is opened.
\end{itemize}
\item Each thread (i.e. process, as Pintos is not multithreaded) has a list of \texttt{proc\_files} to represent the file descriptors it has open. Two different \texttt{proc\_files} (even open in the same process) can have the same file member, but a different fd, due to it being opened twice. 
\item A lock used to ensure synchronization when accessing or modifying files.

\end{enumerate}


\subsubsection{File descriptors}
Each open file in a process has a file descriptor that represents the file. In other words, file descriptors index the open files in a process. In our implementation, we have a \texttt{proc\_file} that has a file descriptor and the associated file. When we are given a file descriptor to access a file, we iterate over the list of \texttt{proc\_file} a thread owns stopping when the file descriptors match.
File descriptors are unique within a single process, but not within the entire OS since they only index open files within a process, i.e, two different processes may have same file descriptors but they work independently.

\subsection{Algorithms}

\subsubsection{Reading and writing user data}

Before entering the specific read or write system call handler, we start by first acquiring the file descriptor, the buffer, and the size to be modified from the stack. Each item we get (fd, buffer, or size) from the stack is first checked to ensure it is a valid pointer from the user and not from the kernel. If any of these items is not a legal pointer, we exit the system with error code -1.  If the pointers are legal, we then call the function \lstinline{sys_read(fd, buffer, size)} or \lstinline{sys_write(fd, buffer, size)} that returns the number of bytes read/written and we store that value in the frame's \texttt{eax}. In detail, this is what happens in each handler:

\begin{enumerate}
\item \lstinline{sys_read()}: We first acquire a lock to ensure synchronization and check irregularities, i.e, the buffer must be a valid buffer, the fd must be a valid fd, and lastly check if the fd is 1, as this is reserved to write to the console, so we immediately exit. If not, two courses of action can occur: 
\begin{enumerate}
\item[a)] The file descriptor given is 0. In that case it means we are trying to read from \texttt{stdin}, so we create an array \texttt{keys}, keep reading one by one the keys pressed with \texttt{input\_getc()} and storing them in \texttt{keys}. We finally copy the contents of \texttt{keys} into the buffer, and return \texttt{size}, as this is the number of bytes read.
\item[b)]  The file descriptor is above 1. In that case we get the file with the file descriptor \lstinline{get_file(fd)}. If no file is found, we release the lock and return -1, meaning the file could not be read. If a file is found, we then call \lstinline{file_read(f, buffer, size)}, storing in a local variable \lstinline{bytes} the number of bytes read, we release the lock, and simply return \lstinline{bytes}.
\end{enumerate}
\item \lstinline{sys_write()}: Like in read, we first acquire a lock to ensure synchronization and check irregularities i.e, the buffer must be a valid buffer, the fd must be a valid fd, and lastly check if the fd is 0, as this is reserved to read from the console, so we immediately exit. If not, again two courses of action can occur: 
\begin{enumerate}
\item[a)] The file descriptor given is 1. In that case it means we are trying to write to \lstinline{stdout}. If we are writing a fairly large amount of bytes to \lstinline{stdout}, we write \lstinline{MAX_CONSOLE_WRITE} bytes per call to \lstinline{putbuf()}, and then write the rest of the bytes, calling \lstinline{sys_write()} recursively. When we're done, we return the size given.
\item[b)]  The file descriptor is above 1. In that case we get the file with the file descriptor \lstinline{get_file(fd)}. If no file is found, we release the lock and return -1, meaning the file could not be read. If a file is found, we then call \lstinline{file_write(f, buffer, size)}, storing in a local variable \lstinline{bytes} the number of bytes read, we release the lock, and simply return \lstinline{bytes}.
\end{enumerate}
\end{enumerate}

\subsubsection{Inspections of the page table}
The least number of inspections would be 1, if all the data being transferred from user space to kernel was stored in a single page. The maximum number of inspections would be 4096, if each byte was stored in a different single page.  For a system call that only copies 2 bytes of data the least number of inspections would be 1, if both bytes were in the same page. The maximum number of inspections would then be 2, if again each byte was stored in a separate page. There is room for improvement if we ensure that bytes copied from the user space are stored in contiguous blocks of memory. Take the previous example: before the maximum was 4096 inspections, but with the proposed improvement the maximum would be 2, which is 4094 inspections less (a big improvement). As with everything, there are some drawbacks to this proposition, as now the storage space is used inefficiently, most likely reducing capacity and performance.

\subsubsection{Wait system call}

We implemented the wait system call, \texttt{sys\_wait}, in terms of \texttt{process\_wait}. \texttt{process\_wait(tid\_t child\_tid)} first works by checking that the \texttt{child\_tid} is indeed a child of the calling process, where the calling process' thread is found using \texttt{thread\_current()}. We can check this by going through the current thread's list of children threads, \texttt{struct list children}. When we find a thread in this list with the same tid as \texttt{child\_tid}, we can break and stop going through this list of children, and set the local variable \texttt{struct thread *child} to this child thread found in the list of children of the current thread. If we finish going through this list of children, then \texttt{child} will be \texttt{NULL}, which it was initialised to, and we can return -1 if this is the case, without ever waiting, since the given tid is not the tid of a child of the calling process' thread. When we check that \texttt{child == NULL}, we can also check the \texttt{bool waited\_on} member of \texttt{child}. \texttt{waited\_on} is initialised to false when the thread is created, and set to true in \texttt{process\_wait()} after this check to see if \texttt{child == NULL || child->waited\_on} holds, so we can return -1. \texttt{waited\_on} therefore is true if \texttt{process\_wait()} has ever been called on this thread. This means we can return -1 from \texttt{process\_wait()} if wait has already been called on \texttt{child\_tid}, without ever waiting. After this, we can set \texttt{child->waited\_on} to true, as mentioned earlier. This is never set to false again, so that we can never call \texttt{wait} on this thread again. Next, a semaphore in \texttt{struct thread *child}, called \texttt{exit\_sema}, is used to wait for \texttt{child} to terminate, by calling \texttt{sema\_down()}. This semaphore is initialised to 0 when the thread is created. \texttt{sema\_up()} is called on a thread's \texttt{exit\_sema} in \texttt{thread\_exit()}, which will cause \texttt{process\_wait()} to stop waiting, as the thread has terminated. Now, \texttt{process\_wait()} can simply return \texttt{child->exit\_status}. Since we changed \texttt{page\_fault()} to set the thread's \texttt{exit\_status} to -1, returning \texttt{child->exit\_status} from \texttt{process\_wait()} will also return -1 if the child was terminated by the kernel. We set this exit status to \texttt{int ret}, so that in between assigning this value to \texttt{ret} and \texttt{return ret}, we can call \texttt{sema\_up(\&child->before\_exit\_sema)}. This semaphore is used to ensure that a thread cannot exit until we have read its exit status in \texttt{process\_wait()}, by calling \texttt{sema\_down()} on a thread's \texttt{before\_exit\_sema} only near the end of \texttt{thread\_exit()}.

\subsubsection{Error handling}

In our design, we have created a function  \lstinline{get_word_on_stack(frame, offset)} common to all system calls that deals with the process of checking for invalid pointers $and$ returning the appropiate word. We believe this encapsulates the error handling and abstracts the code of the system call handler into a more clean function. In \lstinline{get_word_on_stack()} we call \lstinline{check_mem_ptr(uaddr)} which is the actual code that checks if the pointer is null and if it is a user address, exiting if an invalid pointer is detected. So then the process of a making a system call is reduced to the following template:

\begin{lstlisting}
int sys_number = get_word_on_stack(frame, 0);  //Checks validity of esp pointer.
switch(sys_number) {
	.
	.
case(SYS_EXAMPLE):
	int word = get_word_on_stack(frame, 1);  //Checks validity of pointers.
	sys_example(word);

}
\end{lstlisting}

This makes the system call handler function more elegant and as noted, abstracts away the process of checking validity of pointers. Finally, we ensure that all allocated resources are freed because when we call \lstinline{sys_exit} we are in turn calling \lstinline{thread_exit()} which takes care of freeing any resources used for that process. In particular, for the file system lock we always ensure we release the lock before returning.

\subsection{Synchronisation}

\subsubsection{Waiting in 'exec' system call}

A semaphore, called \texttt{load\_sema}, in the child thread that is returned by \texttt{process\_execute()} in the \texttt{exec} system call is used to ensure that the new executable has completed loading before the \texttt{exec} system call returns. \texttt{sema\_down()} is called on \texttt{child->load\_sema} in \texttt{sys\_exec()}, which causes \texttt{sys\_exec()} to wait until the child has completed loading, before we check to see whether it has successfully loaded. \texttt{load\_sema} is initialised to 0 when the thread is created. \texttt{sema\_up()} is only called on a thread's \texttt{load\_sema} in \texttt{start\_process()}, for that thread's process, after the executable has been succcessfully loaded (after \texttt{load()} has been called). Each thread has a \texttt{bool loaded} member, which is set to false when the thread is initialised, and is set to the load success/failure status in \texttt{start\_process()} for that thread. This status is returned from the call to \texttt{load()} in \texttt{start\_process()}. If \texttt{child->loaded} is false in \texttt{sys\_exec()} after we have waited for the executable to have completed loading, we can return -1 from \texttt{sys\_exec()}, as the executable has failed to load.

\subsubsection{Race conditions in 'wait' system call}

If P calls wait(C), the child's \texttt{exit\_sema} ensures that P will wait for C to terminate, as \texttt{sema\_down()} is called here, when the semaphore is initialised to 0, and \texttt{sema\_up()} is only called in \texttt{thread\_exit()}. The use of a sempahore ensures that if P calls wait(C) before C exits, then P will wait for C to exit. If P calls wait(C) after C has called \texttt{thread\_exit()}, then we use another semaphore to ensure that we can still obtain the child's exit status. The semaphore \texttt{before\_exit\_sema}, in a \texttt{struct thread}, ensures that a thread cannot finish exiting in \texttt{thread\_exit()} until we have read its exit status in \texttt{process\_wait()}. The semaphore is initialised to 0 when a thread is created, \texttt{sema\_down()} is called on this semaphore in \texttt{thread\_exit()} to make it wait for \texttt{sema\_up()} to have been called in \texttt{process\_wait()}, which is called just after we grab that thread's exit status (This thread is the \textit{child} thread in \texttt{process\_wait()}). Since all resources are freed for a process during \texttt{process\_exit()}, which is called whenever a process terminates, resources are always freed in both cases.

\subsection{Rationale}

\subsubsection{Accessing user memory}

The function \texttt{static uint32\_t get\_word\_on\_stack(struct intr\_frame *f, int offset)} in \texttt{syscall.c} is used to access user memory in the interrupt stack frame \texttt{f}. The \texttt{intr\_frame}'s saved stack pointer, \texttt{f->esp}, is set to the value of \texttt{uint32\_t *esp}, so that we can return a 4-byte word from this stack pointer at a given \texttt{offset}. Before returning \texttt{*(esp + offset)} as the word required from user memory, we check that pointer \texttt{(esp + offset)} is valid, using the function \texttt{check\_mem\_ptr(const void *uaddr)}. If the pointer is invalid, then \texttt{check\_mem\_ptr()} will terminate the current process with an exit status of -1, because the memory at this address should not be accessed. The pointer is invalid if it is \texttt{NULL}, if it is not in the user address space, or it is an unmapped virtual address. This means that to access the system call number in \texttt{syscall\_handler()}, we can call \texttt{get\_word\_on\_stack(f, 0)}, and to get the exit status out of user memory in the case of an exit system call, we can call \texttt{get\_word\_on\_stack(f, 1)}. The advantages of implementing access to user memory this way is that the pointer is checked on every call to \texttt{get\_word\_on\_stack()}, so any invalid pointer will be rejected and the offending process will be terminated. Also, this allows us to ensure resources are not leaked, because we can just return an error code by calling \texttt{sys\_exit(-1)}, which frees all resources, which would be more difficult if an invalid pointer caused a page fault, because we couldn't return an error code from the memory access.

\subsubsection{File descriptors}

We implemented file descriptors by giving each \texttt{struct thread} an \texttt{int next\_file\_descriptor} member. This member is initialised to 2 when a thread is initialised. When a file is opened using the open system call, that file is given the current value of \texttt{next\_file\_descriptor} in the calling process' \texttt{struct thread}, and then that thread's \texttt{next\_file\_descriptor} is incremented, so that the next file to be opened by this process will have a unique file descriptor. \texttt{next\_file\_descriptor} is initialised to 2, because 0 and 1 are reserved for \texttt{stdout} and \texttt{stdin}, respectively, and all file descriptors will be non-negative. This means that it is very easy to check that a file descriptor is valid during a system call, by checking that it is non-negative and is not larger than or equal to the current process' \texttt{next\_file\_descriptor}. This design for file descriptors makes it very easy to ensure all file descriptors in a process are unique. Each file is stored as a \texttt{struct proc\_file}, which has an \texttt{int fd} member, as well as a \texttt{struct file *file} member, which means that the same file can be opened twice by the same process, yet have different file descriptors. Clearly following from this, the same file opened by two different processes will have unrelated file descriptors. A disadvantage to this design is that if a file is closed, its file descriptor is not used by another file opened in the same process, meaning that it is more difficult to check that we are not closing the file with the same file descriptor twice.

\subsubsection{\texttt{tid\_t} to \texttt{pid\_t} mapping}

We did not change the mapping of \texttt{tid\_t} to \texttt{pid\_t} from the identity mapping. This is because Pintos is not multithreaded, so each process only has a single thread. Since we occasionally need to acquire a thread's \texttt{tid\_t} from a process' \texttt{pid\_t}, and vice versa, it makes sense to keep the identity mapping between these.

%----------------------------------------------------------------------------------------

\end{document}
