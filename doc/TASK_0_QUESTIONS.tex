\documentclass{article}

\usepackage{fancyhdr} % Required for custom headers
\usepackage{lastpage} % Required to determine the last page for the footer
\usepackage{extramarks} % Required for headers and footers
\usepackage[usenames,dvipsnames]{color} % Required for custom colors
\usepackage{graphicx} % Required to insert images
\usepackage{listings} % Required for insertion of code
\usepackage{courier} % Required for the courier font
\renewcommand{\_}{\char`_}

% Margins
\topmargin=-0.45in
\evensidemargin=0in
\oddsidemargin=0in
\textwidth=6.5in
\textheight=9.0in
\headsep=0.25in

\linespread{1.1} % Line spacing

\lstset{language=C,
                basicstyle=\ttfamily,
                keywordstyle=\color{blue}\ttfamily,
                stringstyle=\color{red}\ttfamily,
                commentstyle=\color{Plum}\ttfamily,
                morecomment=[l][\color{magenta}]{\#}
}


% Set up the header and footer
\pagestyle{fancy}
\lhead{Group 18} % Top left header
\chead{Task 0: Alarm Clock} % Top center head
\rhead{\firstxmark} % Top right header
\lfoot{\lastxmark} % Bottom left footer
\rfoot{Page\ \thepage\ of\ \protect\pageref{LastPage}} % Bottom right footer
\renewcommand\headrulewidth{0.4pt} % Size of the header rule
\renewcommand\footrulewidth{0.4pt} % Size of the footer rule

\setlength\parindent{0pt} % Removes all indentation from paragraphs

%----------------------------------------------------------------------------------------
%	TITLE PAGE
%----------------------------------------------------------------------------------------

\title{
\vspace{2in}
\textmd{\textbf{Task 0: Alarm Clock}}\\
\normalsize\vspace{0.1in}\small{Due\ on\ Tuesday,\ January\ 26,\ 2016}\\
\vspace{0.1in}\large{\textbf{Pintos Group 18}}
\vspace{3in}
}

\author{Corentin Herbinet, Ignacio Navarro, Vinothan Shankar, William Springsteen}
\date{}

%----------------------------------------------------------------------------------------

\begin{document}

\maketitle

\newpage


\section{Preliminary Questions}
\subsection{Git}
The Git command that should be run to retrieve a copy of our group's shared 
Pintos repository in our local directory is git clone \texttt{https://gitlab.doc.ic.ac.uk/lab1516\_spring/pintos\_18.git}.
At least one of our members, however, has set up SSH keys and host aliases,
and would use the shorter \texttt{git clone gitlab:lab1516\_spring/pintos\_18.git}.

\subsection{strcpy()}
A call to \texttt{strcpy()} can overflow the buffer reserved for its output string. 
In other words, the function doesn't know the length of the destination string,
so that if the string to be copied is longer than the reserved memory for 
the destination string, the string will be copied corrupting other memory 
(leading to crashes and potentially become an exploitable system).


\subsection{Time slice and ticks}
The default length of a scheduler time slice in Pintos is 4 ticks, 
which is equivalent to 0.04 seconds, as the timer ticks 100 times per second.

\subsection{Thread system}
A thread can be have one of four states - \texttt{THREAD\_RUNNING}, \texttt{THREAD\_READY}, \texttt{THREAD\_BLOCKED} or \texttt{THREAD\_DYING}. The function \texttt{thread\_create(name, priority, function, argument)} creates a new context to be scheduled, 
passing the function to be run in the context. When the function returns, the thread terminates. 
Just before \texttt{schedule()} is called, the current running thread state must be changed from 
\texttt{THREAD RUNNING}, and interrupts must be disabled. Interrupts will be re-enabled by the next 
thread to be scheduled. Context switching begins when \texttt{schedule()} is called. From here, 
\texttt{running\_thread()} will be called to store a pointer the current running thread in the variable \texttt{cur}, 
and next \texttt{next\_thread\_to\_run()} will be called to store a pointer to the next thread to be run in the
variable \texttt{next}. Before any switching occurs, \texttt{schedule()} asserts that interrupts are disabled by calling 
\texttt{intr\_get\_level()}, and \texttt{is\_thread(next)} is called to assert that \texttt{next} is indeed pointing to a valid thread. 
Now, if the pointers obtained earlier aren’t pointing to the same thread (i.e. the next thread to be run 
is not the thread that is already running), \texttt{switch\_threads(cur, next)} is called. When a thread calls 
\texttt{switch\_threads()}, another thread starts running, and the new thread returns from \texttt{switch\_threads()}, 
returning the previously running thread. \texttt{switch\_threads()} will save registers on the stack, save 
the CPU current stack pointer in the current struct thread’s stack member and restore registers 
from the stack. The pointer returned from \texttt{switch\_threads()} is stored in the variable prev. Finally, 
\texttt{thread\_schedule\_tail(prev)} is called to do the rest of the scheduling. This function will change the 
state of the new thread to \texttt{THREAD\_RUNNING}. Also, if the previous thread was dying, it will be 
destroyed, and the page containing its struct thread and stack will be freed. \texttt{thread\_schedule\_tail()}
is separate from \texttt{schedule()} because the first time a thread is created using \texttt{thread\_create()}, the 
new thread hasn't started running yet, so it cannot be running inside \texttt{switch\_threads()}, which is what 
\texttt{schedule()} expects, so \texttt{thread\_schedule\_tail()} is called directly when creating some fake stack frames. 

\subsection{Lack of reproducibility}
Reproducibility refers to the capacity of a phenomenon, particularly
an experimental result, to be replicated.  In a programming context,
this is of particular importance in testing: if a program does not
behave consistently, no test result can be relied upon, because even
if it passes once it might fail on the next invocation.  For behaviour
to be reproducible it must be clear what determines it (which, of
course, requires that it be deterministic).  Lack of reproducibility
makes debugging particularly difficult as the debugging process must
be run repeatedly until the behaviour to be studied actually occurs -
something which might never happen.

\subsection{Printing unsigned integers}
In Pintos we would first include the \texttt{<stdint.h>} and the \texttt{<inttypes.h>} header file.
The first header would let us create the \texttt{uint64\_t int} and then with the second header
we would \texttt{printf()} the \texttt{int} with the macros \texttt{PRIu64}: \newline
  \texttt{uint64\_t x = ...;} \newline
  \texttt{printf("value=\%"PRIu64"", x);}

\subsection{Semaphores and locks}
Essentially a lock is a subset of a semaphore where the initial value is set to 1. 
So then regarding functions, a lock has a \texttt{lock\_release()} equivalent to a 
semaphore's \texttt{sema\_up()}, a \texttt{lock\_try\_acquire()} equivalent to a 
semaphore's \texttt{sema\_try\_down()} and a \texttt{lock\_acquire()} equivalent to a semaphore's
\texttt{sema\_down()}. 
Regarding data structures, a semaphore struct has a list of waiting threads, 
while a lock struct has a semaphore struct that itself has a list of waiting threads.
The extra property that locks have that semaphores do not is that only the thread that 
acquires a lock, called the lock's owner, is allowed to release it. 

\subsection{Thread's size and limitations}
The sum of the sizes of a struct thread and the associated thread's
execution stack is limited to a single page of memory, 4KiB in size.
Because the kernel stack grows downward from the end of the page, and
the struct thread is located at the beginning, if the stack grows too
large it will corrupt the thread state struct.  This overflow is
detected by examining the final member of the struct, ordinarily set
to the value \texttt{THREAD\_MAGIC}.  If it is not, the stack must have
overflowed and rewritten it.  While it is in principle possible for
the overflowing value to be equal to \texttt{THREAD\_MAGIC}, Pintos does not
handle this (unlikely) case.

\subsection{Output of test's failures}
If the test \texttt{src/tests/devices/alarm-multiple} fails, we would find its output 
log in the file \newline \texttt{src/tests/devices/alarm-multiple.output} and its result log in the 
file \newline \texttt{src/tests/devices/alarm-multiple.result}.
\newpage

\section{Design Questions}
\subsection{Data Structures}
\subsubsection{Purpose of new variables}

\begin{enumerate}
 \item \begin{lstlisting}
struct thread
  {
    /* Owned by thread.c. */
    tid_t tid;                          /* Thread identifier. */
    enum thread_status status;          /* Thread state. */
    char name[16];                      /* Name (for debugging purposes). */
    uint8_t *stack;                     /* Saved stack pointer. */
    int priority;                       /* Priority. */
    struct list_elem allelem;           /* List element for all threads list. */

    /* Shared between thread.c and synch.c. */
    struct list_elem elem;              /* List element. */

    struct semaphore timer_wait_sema;  //ADDED
    struct list_elem sleep_elem;       //ADDED 
    int64_t ticks_to_wake_on;          //ADDED 

#ifdef USERPROG
    /* Owned by userprog/process.c. */
    uint32_t *pagedir;                  /* Page directory. */
#endif

    /* Owned by thread.c. */
    unsigned magic;                     /* Detects stack overflow. */
  };
\end{lstlisting}

\item \begin{lstlisting}
struct list timer_waiting_threads = LIST_INITIALIZER(timer_waiting_threads);
\end{lstlisting}

\end{enumerate}

\begin{enumerate}
    \item 
    \begin{enumerate}
        \item We added a semaphore \texttt{timer\_wait\_sema} to \texttt{struct thread} to add proper synchronization in case it needs to go to sleep.
        \item We added a \texttt{list\_elem sleep\_elem} for the waiting
        thread list.
        \item We added a \texttt{int64\_t ticks\_to\_wake\_on} to know how many 
        ticks the thread needs to sleep for.
    \end{enumerate}
    \item We added a list \texttt{timer\_waiting\_threads} to keep a list of waiting
    threads in the operating system.
\end{enumerate}



\subsection{Algorithms}
\subsubsection{Call to timer\textunderscore sleep()}

In a call to \texttt{timer\_sleep()}, the following steps are executed:

\begin{enumerate}
    \item We check that the number of ticks given as an integer argument is strictly positive. If it is not, we return straight away as the thread is not meant to sleep.
    \item We calculate the tick number that the thread needs to wake on, depending on the current number of ticks and the argument given to the function.
    \item We insert the thread in an ordered list at the right position, using a function pointer \texttt{less\_wake} that compares the number of ticks needed for two threads to wake up.
    \item Finally, we call the \texttt{sema\_down} method on the thread's semaphore to make it wait until a \texttt{sema\_up} method is called on that thread's semaphore.
\end{enumerate}

On each timer tick, the following actions are performed by the \texttt{timer\_interrupt()} method:

\begin{enumerate}
    \item The number of ticks is incremented.
    \item We traverse the list of waiting threads, removing from the list and calling \texttt{sema\_up} on each thread which has to be woken up.
\end{enumerate}

\subsubsection{Minimizing time spent in timer\textunderscore interrupt()}

To minimise the time spent in \texttt{timer\_interrupt()}, we have decided to use an ordered list of threads waiting to be woken up. This allows us to only iterate over the list until a thread whose number of ticks needed to be woken up is superior to the current number of ticks. Once this happens, we exit the loop using a break statement. This is more efficient than looping through all the threads using the \texttt{thread\_foreach()} method.

\subsection{Synchronization}
\subsubsection{Avoiding race conditions when multiple threads call timer\textunderscore sleep()}
To avoid race conditions when multiple threads call \texttt{timer\_sleep()} we used a semaphore that each thread owns. When a thread
enters a 
\subsubsection{Avoiding race conditions when timer\textunderscore interrupt() is called within timer\textunderscore sleep()}
To avoid race condition when \texttt{timer\_interrupt()} is called within \texttt{timer\_sleep()} we use a semaphore. So then it doesn't 
matter if \texttt{sema\_up()} is called in \texttt{timer\_interrupt()} before \texttt{sema\_down()} in \texttt{timer\_sleep()} or vice versa.
\subsection{Rationale}

\subsubsection{Choice of design}

A semaphore initialised to 0 is used to make a thread sleep and wake up after the correct amount of ticks, because a semaphore initialised to this value is primarily used to wait for an event that will happen exactly once. This event is the correct amount of ticks having passed since \texttt{timer\_sleep()} was called. This meant that we didn't need to create another data structure to do this task. Another reason that semaphores are useful here is that if there is an interrupt before \texttt{sema\_down()} can be called, but after the thread has been added to the list of waiting threads, which means \texttt{sema\_up()} may be called before \texttt{sema\_down()}, \texttt{timer\_sleep()} will still function correctly. This is because \texttt{sema\_down()} will make the thread wait until the semaphore has a value greater than 0, so if \texttt{sema\_down()} is called after \texttt{sema\_up()}, the thread simply won't wait. This must be correct functionality, as \texttt{sema\_up()} would only be called if the thread was due to wake up.

\subsubsection{Superiority to other designs}

Another design we considered was to use \texttt{thread\_block()} and \texttt{thread\_unblock()} to make the thread sleep and wake up, instead of using a semaphore. This is quite low-level, and semaphores result in a higher level of abstraction in the code. Also, we initially just inserted each new sleeping thread to the front of the list of waiting threads. This meant that during a call to \texttt{timer\_interrupt()}, the whole list had to be traversed, and the function has to check each thread in the list to see if it can be woken up. However, we changed this so that \texttt{list\_insert\_ordered()} is used to insert sleeping threads to the list, where the thread due to wake up first is at the head of the list. This means that if the list is traversed from the head onwards, we can stop traversing the list once we find a thread that can't be woken up yet. This allows less time to be spent in the timer interrupt handler.


%----------------------------------------------------------------------------------------

\end{document}