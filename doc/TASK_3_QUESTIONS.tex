\documentclass{article}

\usepackage{fancyhdr} % Required for custom headers
\usepackage{lastpage} % Required to determine the last page for the footer
\usepackage{extramarks} % Required for headers and footers
\usepackage[usenames,dvipsnames]{color} % Required for custom colors
\usepackage{graphicx} % Required to insert images
\usepackage{listings} % Required for insertion of code
\usepackage{courier} % Required for the courier font
\usepackage{caption}
\usepackage{multirow}
\usepackage{subcaption}
\renewcommand{\_}{\char`_}
\renewcommand{\tt}{\lstinline}

% Margins
\topmargin=-0.45in
\evensidemargin=0in
\oddsidemargin=0in
\textwidth=6.5in
\textheight=9.0in
\headsep=0.25in

\linespread{1.1} % Line spacing

\lstset{language=C,
                basicstyle=\ttfamily,
                keywordstyle=\color{blue}\ttfamily,
                morekeywords={bool},
                stringstyle=\color{red}\ttfamily,
                commentstyle=\color{Plum}\ttfamily,
                morecomment=[l][\color{magenta}]{\#}
}


% Set up the header and footer
\pagestyle{fancy}
\lhead{Group 18} % Top left header
\chead{Task 3: User Memory} % Top center head
\rhead{\firstxmark} % Top right header
\lfoot{\lastxmark} % Bottom left footer
\rfoot{Page\ \thepage\ of\ \protect\pageref{LastPage}} % Bottom right footer
\renewcommand\headrulewidth{0.4pt} % Size of the header rule
\renewcommand\footrulewidth{0.4pt} % Size of the footer rule

\setlength\parindent{0pt} % Removes all indentation from paragraphs

%----------------------------------------------------------------------------------------
%	TITLE PAGE
%----------------------------------------------------------------------------------------

\title{
\vspace{2in}
\textmd{\textbf{Task 3: Virtual Memory}}\\
\normalsize\vspace{0.1in}\small{Due\ on\ Tuesday,\ March\ 22,\ 2016}\\
\vspace{0.1in}\large{\textbf{Pintos Group 18}}
\vspace{3in}
}

\author{Corentin Herbinet, Ignacio Navarro, William Springsteen}
\date{}

%----------------------------------------------------------------------------------------

\begin{document}

\maketitle
\newpage

\section{Page Table Management}

\subsection{Data Structures}

\subsubsection{Purpose of Supplemental Page Table Data Structures}

\subsection{Algorithms}

\subsubsection{Locating Frame for Given Page}

\subsubsection{Coordinating Accessed and Dirty Bits}

\subsection{Synchronisation}

\subsubsection{Race When Getting a New Frame}

\subsection{Rationale}

\subsubsection{Virtual-to-Physical Mappings}

\section{Paging To and From Disk}

\subsection{Data Structures}

\subsubsection{Purpose of Frame Table and Swap Table Data Structures}

\subsection{Algorithms}

\subsubsection{Choosing a Frame to Evict}

\subsubsection{Adjusting Data Structures when Losing a Frame}

\subsubsection{Heuristic for Stack Growth}

\subsection{Synchronisation}

\subsubsection{Basics of VM Synchronisation Design}

\subsubsection{Page Fault Causing Eviction}

\subsubsection{Handling Access to Paged-Out Pages During System Calls}

\subsection{Rationale}

\subsubsection{Number of Locks}

\section{Memory Mapped Files}

\subsection{Data Structures}

\subsubsection{Purpose of File Mapping Table Data Structures}
\begin{enumerate}

\item

\begin{lstlisting}
/* The memory map table will be a hash table mapping a mapid_t to a
   struct mmap_mapping. */
struct mmap_mapping {
  struct hash_elem hash_elem;
  mapid_t mapid; /* Uniquely identifies the mapping (within the process). The
                    hash table mmap_table will take mapid as a key, and the
                    struct mmap_mapping it is in will be the value. */
  int num_pages; /* Number of pages that this file will take up when mapped.
                    If end file is just 1 byte into a page, that whole page is
                    needed. */
  void *start_uaddr; /* Start address that file is mapped to. */
  void *end_uaddr; /* End address that file is mapped to. */
  struct file *file; /* File that is mapped. Not the same struct as in
                        another mmap_mapping for same file, as
                        file_reopen() is used. */
};
\end{lstlisting}

\item

\begin{lstlisting}
struct thread 
 {
    .
    .
    /* Memory mapping members. */
    struct hash mmap_table; /* Mapping between mapid_t and struct mmap_mapping. */
    struct lock mmap_table_lock; /* Acquired/released before/after calling
                                    hash_insert()/hash_delete() on this
                                    threads mmap_table. */
    mapid_t next_mapid; /* Next mmap mapping for this thread will take this as its
                           mapid. Incremented after a new mapping is added.
                           Initially set to 0. */
    .
    .
 };
\end{lstlisting}

\end{enumerate}

\begin{enumerate}

\item

\begin{itemize}
\item
We created a new \texttt{struct}, \texttt{struct mmap\_mapping}, which will be an entry in the memory map table. 
\item
The first member of this \texttt{struct} is a \texttt{hash\_elem} so that this \texttt{struct} can be inserted into the memory map table for a thread.
\item
\texttt{mapid\_t mapid} uniquely identifies the mapping within its process. It's also used to lookup the \texttt{struct mmap\_mapping} that contains it in the mmap hash table. 
\item
\texttt{int num\_pages} is simply the number of pages that the mapped file will take up when it is mapped. \texttt{num\_pages} is needed in \texttt{sys\_munmap()} so that we know how many pages to remove from the process' list of virtual pages, and so we can write the page back to the file. 
\item
\texttt{start\_uaddr} is needed so that we can actually find the start of the file after it has been mapped, such as in \texttt{sys\_munmap} so that we can find the pages to remove from the process' list of virtual pages and write the page back to file. \texttt{end\_uaddr} can be used to work out the length of the file, and is also used to determine how many bytes 'stick out' on the final page.
\item
\texttt{struct file *file} is the actual file that has been mapped to the memory location. It is used to write certain pages back to \texttt{file} in \texttt{sys\_munmap()}.

\end{itemize}

\item

We added 3 members to \texttt{struct thread} for memory mapping. 

\begin{itemize}
\item
The first member is \texttt{mmap\_table} - the actual hash table for the memory mappings. This hash table will map \texttt{mapid}'s to \texttt{struct mmap\_mapping}'s.
\item
\texttt{mmap\_table\_lock} is required so that we can lock before calling \texttt{hash\_insert()} and \texttt{hash\_delete()}.
\item
\texttt{next\_mapid} is the \texttt{mapid} that will be given to the next mapping. It is incremented after each successful mapping, to ensure the uniqueness of a mapping's \texttt{mapid}.
\end{itemize}
s
\end{enumerate}

\subsection{Algorithms}

\subsubsection{Integrating Memory Mapped Files into Virtual Memory Subsystem}

\subsubsection{Overlapping File Mapping}

\subsection{Rationale}

\subsubsection{Sharing Code Between \texttt{mmap} and Data Demand-Paging from Executables}

\end{document}